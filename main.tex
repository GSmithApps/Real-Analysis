\documentclass{article}
\usepackage[utf8]{inputenc}
\usepackage[letterpaper, total={7.5in, 9in}]{geometry}
\title{Intermediate Analysis}
\author{Grant Smith }
\date{Spring 2022}

\begin{document}

\maketitle

\section{Sequence Convergence}

\begin{enumerate}
    \item Prove the convergence of $$\lim_{n\rightarrow \infty} (\sqrt{n^{2}+n}-n)$$
    \begin{enumerate}
        \item First, notice that this equation approaches 0.5 as n increases.  Thus, we will be proving:
        $$\forall \epsilon > 0,  \exists N s.t. \forall n > N, \left| \sqrt{n^{2}+n}-n - \frac{1}{2}\right| < \epsilon$$
        \item which is the same as proving both of the following:
        $$ \sqrt{n^{2}+n}-n - \frac{1}{2} < \epsilon$$
        $$ \sqrt{n^{2}+n}-n - \frac{1}{2} > -\epsilon$$
        \item then add $n + 1/2$ to both sides
        $$ \sqrt{n^{2}+n} < \epsilon + n + \frac{1}{2}$$
        $$ \sqrt{n^{2}+n} > -\epsilon + n + \frac{1}{2}$$
        \item square both sides to get:
        $$ n^{2}+n < \epsilon^2 + 2\epsilon n + 2 \epsilon .5 + n^2 + 2 n .5 + .25$$
        $$ n^{2}+n > \epsilon^2 - 2\epsilon n - 2 \epsilon .5 + n^2 + 2 n .5 + .25 $$
        \item cancel the $n^{2}+n$ from both sides to get
        $$ 0 < \epsilon^2 + 2\epsilon n + 2 \epsilon .5 + 0 .25$$
        $$ 0 > \epsilon^2 - 2\epsilon n - 2 \epsilon .5 + 0 .25 $$
        \item Bring the $2\epsilon n$ to the other side
        $$ -2\epsilon n < \epsilon^2  + 2 \epsilon .5 + 0 .25$$
        $$ 2 \epsilon n > \epsilon^2  - 2 \epsilon .5 + 0 .25 $$   
        \item divide by $2\epsilon$ on the second, and divide by $-2\epsilon$ on the first, which flips the direction of the $<$ to $>$.
        $$ n > \frac{\epsilon^2  + 2 \epsilon .5 + 0 .25}{-2\epsilon} $$
        $$ n > \frac {\epsilon^2  - 2 \epsilon .5 + 0 .25}{2 \epsilon} $$
        \item Given that both of these need to be true, and the second is bigger than the first because the second is positive and the first is negative, we can just use the second requirement. Also, notice that this is a decreasing function in $\epsilon$, as desired.
    \end{enumerate}
    \newpage
    \item Prove the convergence of $$\lim_{n\rightarrow \infty} \left(1+\frac{1}{n}\right)^{3}$$
    \begin{enumerate}
        \item First, note that this sequence approaches 1. Thus, we need to prove that:
            $$\forall \epsilon > 0,  \exists N s.t. \forall n > N, \left| \left(1+\frac{1}{n}\right)^{3} - 1\right| < \epsilon$$
        \item Noting that 
        $$\left(1+\frac{1}{n}\right)^{3} > 1$$
        We can drop the absolute value and get
        $$\left(1+\frac{1}{n}\right)^{3} - 1 < \epsilon$$
        \item Move the 1 to the other side to get:
        $$\left(1+\frac{1}{n}\right)^{3} < 1+ \epsilon$$
        \item Take the cube root of both sides
        $$1+\frac{1}{n} < \left(1+ \epsilon\right) ^{\frac{1}{3}}$$
        \item subtract from to both sides
        $$\frac{1}{n} < \left(1+ \epsilon\right) ^{\frac{1}{3}} -1$$
        \item switch positions of the fractions
        $$\frac{1}{\left(1+ \epsilon\right) ^{\frac{1}{3}} -1} < n$$
        \item Switch sides:
        $$n > \frac{1}{\left(1+ \epsilon\right) ^{\frac{1}{3}} -1}$$
        Which increases as $\epsilon$ decreases
        
    \end{enumerate}
    \newpage
    \item Prove the convergence of: 
    $$\lim_{n\rightarrow \infty} \frac{\sin n}{n^{2}}$$
        \begin{enumerate}
        \item First, note that this sequence approaches 0. Thus, we need to prove that:
            $$\forall \epsilon > 0,  \exists N s.t. \forall n > N, \left|\frac{\sin n}{n^{2}}\right| < \epsilon$$
            \item Noting that $n^2 > 0$, we can rewrite as: 
            $$ \frac{\left|\sin n\right|}{n^{2}} < \epsilon$$
            \item Noting that 
            $$\left|\sin n\right| \leq 1 \rightarrow \frac{\left|\sin n\right|}{n^{2}} \leq \frac{1}{n^2} \leq \frac{1}{n} $$
            \item Which means that if we can prove convergence of $1/n$, then we have proven the convergence of the sequence in question. Thus, we want:
            $$\frac{1}{n} < \epsilon$$
            or
            $$\frac{1}{\epsilon} < n$$
            or
            $$n > \frac{1}{\epsilon}$$

        
    \end{enumerate}
                \newpage
            \item Prove that if $\{b_n\}$ is a sequence of positive terms and $b_n \rightarrow b > 0$, then there is a positive lower bound $m > 0$ such that $b_n \geq m$ for all $n$.
\begin{itemize}
    \item If $b_n \rightarrow b > 0$, then for any $\epsilon$ there exists an $N$ such that for all $n > N$, $|b_n - b| < \epsilon$. In particular, if we let $\epsilon = b/2$, then for all $n > N$, we can ensure that $b_n > b/2$, and since $b > 0$, we know that $b/2 > 0$. 
    \item Next, since there are only a finite number of $b_n$ values before N, we can take the minimum of those values. Call this $m'$.  We know that each term is positive, so $m'$ is positive.
    \item Next, we take the minimum of $b/2$ and $m'$, and we can take half of that value as well, and call it m: $\min  \left\{b/2,m'\right\}/2 = m$. This $m$ is one of infinitely many possible lower bounds.
\end{itemize}
\end{enumerate}



\end{document}
